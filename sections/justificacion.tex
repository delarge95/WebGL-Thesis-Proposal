\section{Justificación}

La elección de Unity WebGL sobre librerías nativas de JavaScript (Three.js, Babylon.js) o soluciones de streaming remoto (Unreal Pixel Streaming) no es arbitraria, sino que responde a requisitos cuantitativos de \textit{Computación de Alto Rendimiento (HPC)} en la web y necesidades del pipeline de arte técnico:

\subsection{Criterios Técnicos de Selección}

\subsubsection{Rendimiento de CPU (WASM vs JavaScript)}

Unity utiliza el backend IL2CPP para transpilar código C\# a C++ y posteriormente a WebAssembly (WASM). Esto permite que la lógica de interacción y los cálculos físicos se ejecuten a velocidad casi nativa, evitando la sobrecarga del \textit{Garbage Collector} de JavaScript que causa micro-congelamientos (frame drops) en aplicaciones complejas. Según documentación oficial de Unity (2024), IL2CPP elimina pauses de GC que en JavaScript pueden exceder 16ms, causando caída por debajo de 60 FPS.

\subsubsection{Pipeline de Assets Robusto}

A diferencia de Three.js que requiere construcción manual de cargadores y gestores de memoria, Unity ofrece un sistema integrado de \textit{Asset Bundles} y compresión de texturas (ASTC/ETC2) esencial para garantizar que el prototipo se mantenga dentro del presupuesto de memoria de 50MB iniciales, permitiendo carga progresiva (Progressive Loading) de assets de alta resolución en segundo plano.

\subsubsection{Optimizaciones Gráficas Avanzadas}

El proyecto requiere técnicas de \textit{Occlusion Culling} (no renderizar geometría bloqueada) y \textit{LOD} (Level of Detail) automático. Implementar esto desde cero en una librería ligera es propenso a errores y consume tiempo de desarrollo que debería dedicarse a la lógica de la aplicación. Unity URP ofrece estas técnicas integradas con profiling visual.

\subsubsection{Herramientas Visuales para Artistas Técnicos}

El flujo de trabajo requiere colaboración entre programadores y artistas técnicos. Unity Shader Graph permite a no-programadores crear materiales PBR complejos, mientras que Timeline facilita la animación del despiece (exploded view) sin código.

\subsubsection{Escalabilidad Industrial}

El uso de un motor comercial estandariza el flujo de trabajo, permitiendo que el prototipo (MVP) sea mantenible y escalable, alineándose con las prácticas de la Industria 4.0. El proyecto puede evolucionar a aplicación de producción sin cambiar stack tecnológico.

\subsection{Aporte a la Ingeniería Multimedia}

Este proyecto aporta al campo de la Ingeniería Multimedia al:
\begin{itemize}
    \item documentar un pipeline técnico replicable para la optimización de activos CAD a WebGL
    \item validar el rol del ingeniero multimedia como puente entre la complejidad técnica y la experiencia de usuario final
    \item generar conocimiento sobre métricas cuantitativas (KPIs) para rendimiento en Web 3D
    \item aplicar teoría cognitiva (Mayer, Paivio) a casos de uso reales de ingeniería
\end{itemize}

\newpage
