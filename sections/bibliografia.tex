\newpage
\section{Referencias Bibliográficas}

\begin{list}{}{%
    \setlength{\leftmargin}{1.27cm}
    \setlength{\itemindent}{-1.27cm}
    \setlength{\itemsep}{0pt}
    \setlength{\parsep}{0pt}
    \setlength{\topsep}{0pt}
    \setlength{\listparindent}{0pt}
    \setlength{\labelwidth}{0pt}
    \setlength{\labelsep}{0pt}
}
\RaggedRight

\item Akenine-Möller, T., Haines, E., \& Hoffman, N. (2018). \textit{Real-Time Rendering} (4th ed.). CRC Press.

\item Bartlett, K. A., \& Dorribo Camba, J. (2023). The Role of a Graphical Interpretation Factor in the Assessment of Spatial Visualization: A Critical Analysis. \textit{Spatial Cognition \& Computation}, \textit{23}(1), 1--30. \url{https://doi.org/10.1080/13875868.2021.1987375}



\item Bowman, D. A., Kruijff, E., LaViola Jr, J. J., \& Poupyrev, I. (2004). \textit{3D User Interfaces: Theory and Practice}. Addison-Wesley.

\item Burley, B. (2012). Physically-Based Shading at Disney. \textit{ACM SIGGRAPH 2012 Course Notes: Practical Physically Based Shading in Film and Game Production}. ACM.

\item Cook, R. L., \& Torrance, K. E. (1982). A reflectance model for computer graphics. \textit{ACM Transactions on Graphics}, \textit{1}(1), 7--24. \url{https://doi.org/10.1145/357290.357293}

\item Cowan, N. (2001). The magical number 4 in short-term memory: A reconsideration of mental storage capacity. \textit{Behavioral and Brain Sciences}, \textit{24}(1), 87--114.

\item Darken, R. P., \& Sibert, J. L. (1996). Wayfinding Strategies and Behaviours in Large Virtual Worlds. \textit{Proceedings of the SIGCHI Conference on Human Factors in Computing Systems (CHI '96)}, 142--149. \url{https://doi.org/10.1145/238386.238459}

\item Fransson, E., Hermansson, J., \& Hu, Y. (2024). A Comparison of Performance on WebGPU and WebGL in the Godot Game Engine. \textit{Proceedings of the 2024 IEEE Conference on Games (CoG)}, 1--8. IEEE. \url{https://doi.org/10.1109/CoG60054.2024.10645582}



\item Hegarty, M. (2004). Mechanical reasoning by mental simulation. \textit{Trends in Cognitive Sciences}, \textit{8}(6), 280--285.

\item Hegarty, M., \& Waller, D. (2004). Spatial Abilities at Different Scales: Individual Differences in Aptitude-Test Performance and Spatial-Layout Learning. \textit{Intelligence}, \textit{32}(2), 151--176.

\item Hevner, A. R., March, S. T., Park, J., \& Ram, S. (2004). Design Science in Information Systems Research. \textit{MIS Quarterly}, \textit{28}(1), 75--105.

\item Hutchins, E. (1995). \textit{Cognition in the Wild}. MIT Press.

\item Karis, B. (2013). Real Shading in Unreal Engine 4. En B. Burley (Ed.), \textit{SIGGRAPH 2013 Course: Physically Based Shading in Theory and Practice}. ACM.

\item Khronos Group. (2011). \textit{WebGL Specification 1.0}. \url{https://www.khronos.org/registry/webgl/specs/latest/1.0/}

\item Köhler, W. (1929). \textit{Gestalt Psychology}. Liveright.

\item Mayer, R. E. (2005). \textit{The Cambridge Handbook of Multimedia Learning}. Cambridge University Press.

\item Mayer, R. E. (Ed.). (2021). \textit{The Cambridge Handbook of Multimedia Learning} (3rd ed.). Cambridge University Press. \url{https://doi.org/10.1017/9781108894333}

\item Miller, G. A. (1956). The Magical Number Seven, Plus or Minus Two: Some Limits on Our Capacity for Processing Information. \textit{Psychological Review}, \textit{63}(2), 81--97.

\item Cabello, R. [mrdoob]. (2024). \textit{three.js - JavaScript 3D Library} [GitHub repository]. GitHub. \url{https://github.com/mrdoob/three.js} (110,000 stars as of November 2024)

\item Nielsen, J. (1994). \textit{Usability Engineering}. Morgan Kaufmann.

\item Nielsen, J. (2000). Why You Only Need to Test with 5 Users. Nielsen Norman Group. \url{https://www.nngroup.com/articles/why-you-only-need-to-test-with-5-users/}

\item Norman, D. A. (2013). \textit{The Design of Everyday Things: Revised and Expanded Edition}. Basic Books.

\item Paivio, A. (1986). \textit{Mental Representations: A Dual Coding Approach}. Oxford University Press.

\item Ries, E. (2011). \textit{The Lean Startup: How Today's Entrepreneurs Use Continuous Innovation to Create Radically Successful Businesses}. Crown Business.

\item Schlick, C. (1994). An Inexpensive BRDF Model for Physically-based Rendering. \textit{Computer Graphics Forum}, \textit{13}(3), 233--246.

\item Sweller, J. (1988). Cognitive load during problem solving: Effects on learning. \textit{Cognitive Science}, \textit{12}(2), 257--285.

\item Sweller, J., van Merriënboer, J. J. G., \& Paas, F. (2019). Cognitive architecture and instructional design: 20 years later. \textit{Educational Psychology Review}, \textit{31}(2), 261--292. \url{https://doi.org/10.1007/s10648-019-09465-5}

\item Unity Technologies. (2021). \textit{Advances in Real-Time Rendering in Games: Part I}. ACM SIGGRAPH 2021 Courses. ACM. \url{https://doi.org/10.1145/3450508.3464571}

\item Unity Technologies. (2024). \textit{Memory in WebGL}. Unity Documentation. \url{https://docs.unity3d.com/Manual/webgl-memory.html}

\item Unity Technologies. (2024). \textit{WebGL: Interacting with browser scripting}. Unity Documentation. \url{https://docs.unity3d.com/Manual/webgl-interactingwithbrowserscripting.html}

\item Walter, B., Marschner, S. R., Li, H., \& Torrance, K. E. (2007). Microfacet Models for Refraction through Rough Surfaces. \textit{Proceedings of the 18th Eurographics Conference on Rendering Techniques}, 195--206.

\item Ware, C. (2012). \textit{Information Visualization: Perception for Design} (3rd ed.). Morgan Kaufmann.

\item Wertheimer, M. (1923). Untersuchungen zur Lehre von der Gestalt II. \textit{Psychologische Forschung}, \textit{4}(1), 301--350.

\item Yu, G., Liu, C., Fang, T., Jia, J., Lin, E., He, Y., Fu, S., Wang, L., Wei, L., \& Huang, Q. (2023). A survey of real-time rendering on Web3D application. \textit{Virtual Reality Intelligent Hardware}, \textit{5}(5), 390--394. \url{https://doi.org/10.1016/j.vrih.2022.04.002}

\end{list}
