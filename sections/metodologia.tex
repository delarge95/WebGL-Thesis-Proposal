\section{Metodología}

Se utilizará una metodología de Investigación Aplicada con un enfoque de Design Science Research (DSR) (Hevner et al., 2004), estructurada en desarrollo Ágil (Scrum Adaptado) con Sprints de 4 semanas.

\subsection{Framework de Design Science Research}

El proyecto sigue el framework DSR de 6 etapas (Hevner et al., 2004):
\begin{enumerate}
    \item Problem Identification: Brecha de información técnica en medios 2D.
    \item Objectives: Prototipo que valida viabilidad de WebGL para visualización de hardware.
    \item Design \& Development: Implementación en Unity siguiendo principios de HCI.
    \item Demonstration: Despliegue online accesible.
    \item Evaluation: Pruebas de usabilidad (SUS), profiling de rendimiento.
    \item Communication: Documentación de pipeline de optimización.
\end{enumerate}

\subsection{Desarrollo en Sprints}

\subsubsection{Sprint 1: Análisis y Fundamentación Técnica (Semana 1-4)}
\begin{itemize}
    \item Selección del modelo CAD de referencia (Dron de Alta Gama).
    \item Benchmarking: Pruebas de rendimiento en dispositivos objetivo (Snapdragon 7 series).
    \item Definición de KPIs: Polígonos ($<$50,000), Draw Calls ($<$50), VRAM ($<$150MB).
\end{itemize}

\subsubsection{Sprint 2: Pipeline de Arte Técnico (Semana 5-8)}
\begin{itemize}
    \item Retopología: Conversión de mallas CAD (NURBS) a polígonos optimizados en Blender.
    \item UV Mapping: Organización de islas UV para maximizar densidad de texel.
    \item Baking: Transferencia de detalles de alta frecuencia a mapas de normales.
\end{itemize}

\subsubsection{Sprint 3: Ingeniería e Implementación (Semana 9-16)}
\begin{itemize}
    \item Configuración de Unity URP con soporte WebGL 2.0.
    \item Programación de scripts C\# (compilados a WASM) para sistema de ``Despiece'' (Exploded View).
    \item Implementación de UI responsiva.
\end{itemize}

\subsubsection{Sprint 4: Validación y Despliegue (Semana 17-24)}
\begin{itemize}
    \item Profiling: Uso de Unity Memory Profiler para detectar fugas de memoria.
    \item Pruebas de Usuario: Evaluación mediante cuestionarios SUS (System Usability Scale).
    \item Despliegue: Publicación en servidor con compresión Gzip/Brotli.
\end{itemize}

\subsection{Tamaño de Muestra para Pruebas de Usabilidad}

Nielsen (2000) demostró que 5 usuarios detectan $\sim$85\% de problemas de usabilidad. Para este proyecto:
\begin{itemize}
    \item Muestra: 8-12 usuarios (target: ingenieros/técnicos).
    \item Justificación: Supera el mínimo de Nielsen, permitiendo detección de issues menos frecuentes.
    \item Protocolo: Think-Aloud + SUS Questionnaire.
\end{itemize}

\subsection{KPIs Técnicos (Cuantitativos)}

\begin{enumerate}
    \item Polygon Budget: $\sum P_i < 50,000$ triángulos.
    \item Texel Density: $\rho = 10.24 \pm 2$ px/cm (±20\% tolerancia).
    \item Draw Calls: $D < 50$ (GPU Instancing + Static Batching).
    \item Frame Time: $T_{frame} < 33.33$ms (30 FPS) en Snapdragon 7 Gen 1.
    \item Memory: Heap Usage $<$ 150MB (medido via Unity Profiler).
    \item Load Time: TTI (Time to Interactive) $<$ 5s en 4G (10 Mbps).
\end{enumerate}

\newpage
