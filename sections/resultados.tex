\section{Resultados o Productos Esperados}

\begin{table}[h!]
\caption{Resultados o Productos Esperados}
\centering
\small
\begin{tabular}{p{5.5cm}p{5cm}p{3.5cm}}
\hline
\textbf{RESULTADO/PRODUCTO} & \textbf{INDICADOR} & \textbf{BENEFICIARIO} \\ \hline
\textbf{1. Prototipo de Software WebGL Funcional} & - URL pública del prototipo. \newline - Implementación verificada de 3+ características interactivas. \newline - Cumplimiento de KPIs ($<$100K polígonos, $>$30 FPS). & - Comunidad Académica. \newline - Industria del Hardware. \newline - Estudiante. \\ 
\textbf{2. Conjunto de Modelos 3D Optimizados} & - Reducción tamaño archivo $\geq$ 90\%. \newline - Archivos .glb disponibles. \newline - Documentación de texel density. & - Plataforma web. \newline - Estudiantes de Multimedia. \\ 
\textbf{3. Documento de Trabajo de Grado} & - Documento final aprobado. \newline - Cumplimiento normas APA 7 UNAD. \newline - \textit{Pipeline} replicable documentado. & - Estudiante. \newline - UNAD. \newline - Comunidad Académica. \\ 
\textbf{4. Informe de Evaluación de Usabilidad y Carga Cognitiva} & - Informe entregado. \newline - Resultados SUS (Usabilidad) y NASA-TLX (Carga Cognitiva). \newline - Comparativa de eficiencia vs medios 2D. & - Estudiante. \newline - Jurados. \\ \hline
\end{tabular}
\par\vspace{0.5\baselineskip}
\textit{Nota}. Los indicadores buscan medir la consecución de los resultados clave del proyecto. \textit{Fuente}. Autoría propia.
\end{table}

\newpage
