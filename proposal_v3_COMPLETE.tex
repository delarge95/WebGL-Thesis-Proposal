\documentclass[12pt,letterpaper]{article}

% --- PACKAGES (Following Reference Document Format) ---
\usepackage[utf8]{inputenc}
\usepackage[spanish]{babel}
\usepackage{times}
\usepackage[margin=2.54cm]{geometry}
\usepackage{setspace}
\usepackage{graphicx}
\usepackage{booktabs}
\usepackage{caption}
\usepackage{amsmath}
\usepackage{amsfonts}
\usepackage{amssymb}
\usepackage[hyphens]{url}
\usepackage[hidelinks]{hyperref}
\usepackage{ragged2e}
\usepackage{indentfirst}
\usepackage{titlesec}
\usepackage{tocloft}

% --- Spanish Labels ---
\addto\captionsspanish{\renewcommand{\tablename}{Tabla}}
\addto\captionsspanish{\renewcommand{\contentsname}{Tabla de Contenido}}

% --- APA 7 UNAD FORMATTING ---
\setstretch{2}
\setlength{\parindent}{1.27cm}

% Page Numbering (Top Right)
\usepackage{fancyhdr}
\pagestyle{fancy}
\fancyhf{}
\renewcommand{\headrulewidth}{0pt}
\fancyhead[R]{\thepage}
\fancypagestyle{plain}{%
  \fancyhf{}%
  \renewcommand{\headrulewidth}{0pt}%
  \fancyhead[R]{\thepage}%
}

% --- ToC: No Section Numbers ---
\renewcommand{\cftsecpresnum}{\relax}
\renewcommand{\cftsecnumwidth}{0pt}
\renewcommand{\cftsubsecpresnum}{\relax}
\renewcommand{\cftsubsecnumwidth}{0pt}
\renewcommand{\cftsubsubsecpresnum}{\relax}
\renewcommand{\cftsubsubsecnumwidth}{0pt}

% ToC Title Centered
\setlength{\cftbeforetoctitleskip}{0\baselineskip}
\setlength{\cftaftertoctitleskip}{0\baselineskip}
\renewcommand{\cfttoctitlefont}{\hfill\normalfont\bfseries\normalsize}
\renewcommand{\cftaftertoctitle}{\null\hfill}

% ToC Entry Fonts (No Bold)
\renewcommand{\cftsecfont}{\normalfont}
\renewcommand{\cftsecpagefont}{\normalfont}
\renewcommand{\cftsubsecfont}{\normalfont}
\renewcommand{\cftsubsecpagefont}{\normalfont}
\renewcommand{\cftsubsubsecfont}{\normalfont}
\renewcommand{\cftsubsubsecpagefont}{\normalfont}

% ToC Leaders (dots)
\renewcommand{\cftsecleader}{\cftdotfill{\cftdotsep}}
\renewcommand{\cftsubsecleader}{\cftdotfill{\cftdotsep}}
\renewcommand{\cftsubsubsecleader}{\cftdotfill{\cftdotsep}}

% ToC Indentation
\setlength{\cftsecindent}{0em}
\setlength{\cftsubsecindent}{1.5em}
\setlength{\cftsubsubsecindent}{3em}

\RaggedRight

% Section Title Formatting
\titleformat{\section}{\normalfont\bfseries\centering}{}{0em}{}
\titlespacing*{\section}{0pt}{0\baselineskip}{0\baselineskip}
\titleformat{\subsection}{\normalfont\bfseries}{}{0em}{}
\titlespacing*{\subsection}{0pt}{0\baselineskip}{0\baselineskip}
\titleformat{\subsubsection}{\normalfont\bfseries\itshape}{}{0em}{}
\titlespacing*{\subsubsection}{0pt}{0\baselineskip}{0\baselineskip}

% --- BEGIN DOCUMENT ---
\begin{document}

% === TITLE PAGE ===
\begin{titlepage}
    \centering
    {\normalfont\bfseries Diseño y Desarrollo de un Prototipo Web 3D Interactivo para la Visualización Técnica y Análisis Estructural de Hardware de Alto Rendimiento mediante Pipelines de Optimización Gráfica y WebAssembly\par}
    \vspace{2\baselineskip}\vspace{2\baselineskip}
    {\normalfont Estudiante\par}{\normalfont Alexander Woodcock Salomón\par}
    \vspace{2\baselineskip}\vspace{2\baselineskip}
    {\normalfont Asesor\par}{\normalfont Javier Gerardo Reina Granados\par}
    \vfill
    {\normalfont Universidad Nacional Abierta y a Distancia - UNAD\par Escuela de Ciencias Básicas, Tecnología e Ingeniería\par Ingeniería Multimedia\par}
    \vspace{1\baselineskip}{2025\par}
\end{titlepage}

\pagestyle{fancy}

% === RESUMEN ===
\section*{Resumen}
\setlength{\parindent}{0pt}
El diseño de hardware de alto rendimiento (drones, sistemas robóticos) enfrenta un desafío crítico en la comunicación técnica: los medios digitales tradicionales (imágenes 2D, PDFs estáticos) imponen alta carga cognitiva para comprender estructuras tridimensionales complejas. Este proyecto aborda esta brecha mediante el diseño, desarrollo y evaluación de un prototipo web 3D interactivo basado en Unity WebGL que optimiza la visualización técnica de hardware mediante pipelines de optimización gráfica avanzados y WebAssembly. El marco teórico integra teoría de carga cognitiva (Mayer, Paivio), interacción humano-computador en 3D (Bowman, Norman), y renderizado físicamente basado (Cook-Torrance, Burley). La metodología sigue Design Science Research con desarrollo iterativo en Sprints de 4 semanas, implementando técnicas de retopología, baking de normales, y Asset Bundles para cumplir con KPIs cuantitativos (< 50,000 polígonos, > 30 FPS en móvil mid-range, < 5s tiempo de carga). La validación incluye profiling de rendimiento (Unity Profiler) y pruebas de usabilidad (SUS, N=8-12 ingenieros/técnicos). El producto final es un MVP científico que valida la viabilidad técnica de WebGL para visualización de ingeniería, estableciendo un pipeline replicable documentado según estándares académicos.
\setlength{\parindent}{1.27cm}

\textit{\textbf{Palabras clave:}} WebGL, Unity, WebAssembly, Renderizado Físicamente Basado (PBR), Visualización 3D Interactiva, Optimización de Assets 3D, Carga Cognitiva, Interacción Humano-Computador, Technical Art, Hardware de Alto Rendimiento.

\newpage

% === ABSTRACT ===
\section*{Abstract}
\setlength{\parindent}{0pt}
High-performance hardware design (drones, robotic systems) faces a critical challenge in technical communication: traditional digital media (2D images, static PDFs) impose high cognitive load for understanding complex three-dimensional structures. This project addresses this gap through the design, development, and evaluation of an interactive 3D web prototype based on Unity WebGL that optimizes technical hardware visualization through advanced graphics optimization pipelines and WebAssembly. The theoretical framework integrates cognitive load theory (Mayer, Paivio), 3D human-computer interaction (Bowman, Norman), and physically-based rendering (Cook-Torrance, Burley). The methodology follows Design Science Research with iterative development in 4-week Sprints, implementing retopology, normal baking, and Asset Bundles to meet quantitative KPIs (< 50,000 polygons, > 30 FPS on mid-range mobile, < 5s load time). Validation includes performance profiling (Unity Profiler) and usability testing (SUS, N=8-12 engineers/technicians). The final product is a scientific MVP that validates the technical feasibility of WebGL for engineering visualization, establishing a replicable, academically documented pipeline.
\setlength{\parindent}{1.27cm}

\textit{\textbf{Keywords:}} WebGL, Unity, WebAssembly, Physically-Based Rendering (PBR), Interactive 3D Visualization, 3D Asset Optimization, Cognitive Load, Human-Computer Interaction, Technical Art, High-Performance Hardware.

\newpage

% === TABLE OF CONTENTS ===
\tableofcontents
\newpage

% DUE TO FILE SIZE, CONTINUING IN NEXT RESPONSE...
